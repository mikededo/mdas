\documentclass[a4paper, 11pt, oneside]{article}

\usepackage[english]{babel}
\usepackage[utf8]{inputenc}
\usepackage[pagestyles]{titlesec}
\usepackage{titletoc}
\usepackage{enumitem}
\usepackage{fancyhdr}
\usepackage[a4paper, margin = 2.5cm, headheight=14pt]{geometry}
\usepackage[hidelinks, colorlinks = false]{hyperref}
\usepackage[bottom]{footmisc}
\usepackage{bookmark}
\usepackage{fontspec}
\usepackage{graphicx}
\usepackage[labelfont = bf, hypcap = false]{caption}
\usepackage{etoolbox}
\usepackage{tabularx}
\usepackage{colortbl}
\usepackage{float}
\usepackage[titles]{tocloft}
\usepackage[table]{xcolor}
\usepackage{url}
\usepackage{subfiles}

\date{}

\title{
\textbf{\huge Metodolgías Ágiles}
\\[8pt]
{Práctica final}
}
\author{Miquel de Domingo i Giralt}

\setmainfont{Helvetica Neue}[BoldFont = Helvetica Neue Bold]

% Variable with the page that keeps the footer style
\def\footerstyle{- \fontsize{10pt}{10pt}\selectfont\thepage\ -}
\pagestyle{fancy}
% Remove the horizontal bar from the header
\renewcommand{\headrulewidth}{0pt}
% Clear everythihg
\fancyhf{}
% Set header style
\rhead{\small\textit{\nouppercase{\rightmark}}}
% Set the footer page number
\fancyfoot[C]{\footerstyle}
% Update the footer in chapter and other plain views
\fancypagestyle{plain}{%
	\renewcommand{\headrulewidth}{0pt}%
	\fancyhf{}%
	\fancyfoot[C]{\footerstyle}%
}

\setlength\parindent{0pt}

\titlespacing{\chapter}{0pt}{-32pt}{12pt}

\begin{document}
\maketitle
\newpage
\section{Problema}
Acudes a una organización porque puedes aportar tu trabajo como Scrum Master.
Hay un equipo trabajando con Scrum desde hace un par de meses, y por lo que te
cuentan parece que no haya mejorado desde la situación anterior.
\\[8pt]
Se introdujo Scrum como proceso para ganar en visibilidad sobre el proyecto,
pensando que aceleraría el proceso de desarrollo, pero sin pensar en los cambios
culturales que se debían introducir a la vez. El Scrum Master nombrado seguía en
funciones de jefe de proyecto, y la mayoría de la gente no se planteaba la
auto-organización como un objetivo deseable todavía, pues estaban cómodas en su
función de acatar órdenes.
\\[8pt]
Además, la desconfianza existente había creado pesados procesos manuales de
aprobaciones de requisitos y de su verificación, para trabajar con el cliente
interno que realiza la toma de decisiones del producto.
\\[8pt]
Se te ofrece trabajar como Scrum Master del equipo, y el actual Scrum Master
pasa a ser Product Owner, pues es quien mejor conoce el producto desarrollado.
\\[8pt]
De las 6 personas del equipo, existían conflictos con dos de ellas y el antiguo
SM, pues rivalizaban por el poder jerárquico dentro de la organización que tenía
el primero. Ahora se encuentran molestas, pues ven que las metodologías ágiles
diluyen esa posición que buscaban.
\\[8pt]
Los responsables de la organización tienen las miradas puestas en este equipo
porque el proyecto es estratégico, y apoyan las iniciativas de cambio, pero no
comprenden todavía el alcance de un cambio cultural basado en los principios
ágiles.
\begin{enumerate}
	\item Descríbeme las ventajas y desventajas de ser nombrado superior
	      jerárquico en ese equipo, para hacer de Scrum Master, y de no serlo.
	\item ¿Alrededor de qué giraría tu primera reunión de coaching individual
	      con cada miembro del equipo?
	\item ¿Cómo podrías crear objetivos globales del equipo?
\end{enumerate}
\newpage
\section{Solución}
\subsection{Problema 1}
Entrar en un nuevo equipo es siempre un proceso complicado y en función de la
unión del equipo puede hacerse bastante tedioso. Además, en un entorno en el que
el ambiente de equipo no es el idóneo, este proceso puede complicarse aún más.
\\[8pt]
En primera instancia, destacaría las siguientes desventajas (siempre es más
fácil ver los problemas). De entrada voy a ser juzgado por las soluciones que
proponga, y en función de la toxicidad de los compañeros que me ven como su
rival, estos pueden llegar a dificultar la prueba de las propuestas. También, de
las personas nuevas se esperan cambios radicales, y, en la sociedad de hoy en
día, se esperan cambios a corto plazo. Desde mi punto de vista, será interesante
concienciar a los integrantes del grupo de la necesidad de cambio tanto
individual como en conjunto, y que en función de la situación actual, quizás sea
un proceso más lento.
\\
Otro punto a considerarse como desventaja es el hecho de tener un enfoque
demasiado autoritario. No dejar que los miembros del equipo sean capaces de
tomar sus propias decisiones, analizar los resultados y luego proponer
soluciones, va a debilitar su motivación así como su sensación de impacto.
Considero que la posición que debería tomar como SM es más de acompañamiento o
consejero, ayudando a ampliar el punto de vista. En este sentido, suponiendo que
tengo cierta experiencia en el mundo de SM, seguramente seré capaz de tener en
cuenta detalles que se les pueden escapar al equipo. De este modo, el equipo
será capaz de tomar sus propias decisiones, con la mi ayuda, reduciendo de ese
modo el autoritarismo.
\\[8pt]
A nivel de ventajas, creo que puede llegar a ser interesante mantener la
jerarquia porqué voy a ser visto con más poder a la hora de resolución de
problemas y organización. Al ser uno de los superiores del equipo, voy a tener
cierto impacto sobre la resolución de problemas con el equipo o individuales.
\\
Quizás el más importante es el impacto que voy a poder tener, siempre y cuando
sea respetado como SM, para poder facilitar el cambio cultural. Voy a poder
tener más capacidad de establecer cambios culturales, ya que tendré más
influencia sobre los demás integrantes del grupo. No hace falta decir que ambas
partes deben respetarse mutuamente, ya que de otra forma solo va a incrementar
el malestar general a nivel de equipo.
\\[8pt]
Aún así, considero que el SM no debería verse como una posición superior en la
jerarquia, de la misma forma que se puede ver el rol the Team Lead. De ser por
mi, fomentaría una visión más horizontal del equipo y de sus miembros. También,
intentaría que se fomentara el feedback y la mejora contínua, con la
colaboración de todos los miembros del equipo. Si un miembro ve que sus
propuestas o decisiones tienen impacto tanto a corto como a largo plazo, va a
sentirse más integrado.
\\
Entrando más a mis experiencias como desarrollador dentro de un equipo, creo que
quizás el punto que más debe evitar un SM es el de intentar ser autoritario.
Desmotiva bastante al personal y directamente bloqueas la capacidad de
cooperación y colaboración que estos puedan llegar a tener tanto como equipo,
como con el SM.
\\
En conclusión, el SM no debería presentarse como una posición superior y
autoritaria, sino como otro rol en el equipo con diferentes funciones.
\subsection{Problema 2}
Tal y como he comentado en el último párrafo del primer problema, es importante
destacar que el SM tendrá diferentes funciones a la de los desarrolladores (no
está especificado en el problema, pero he supuesto que se trata de un equipo de
desarrollo, pues es el área en la que trabajo). De la misma forma que un miembro
o miembra sénior de la parte de frontend va a tener su poder de decisión y
respeto en ciertos aspectos (por ejemplo, en el refinement de tíquets
relacionados con el FE), yo como SM voy a tener unos otros. Pues el principal
objetivo sería determinar que es lo que cada desarrollador o desarrolladora
espera de mi persona. Además, va a ser igual de importante determinar el rol de
los integrantes, para poder detectar en que partes carece el equipo. La detección
de dichas carencias va a servirme para poder determinar otras áreas de impacto
para los integrantes. De esta forma, si un integrate quiere asumir más
responsabilidad dentro del equipo, puede llegar a cubrir dicha carencia.
\\
Si el equipo lleva tiempo con conflictos, será importante que cada miembro, en
un entorno aislado y sin juicios, pueda comentar los problemas que ve
actualmente. Además, se les debe brindar la oportunidad de proponer mejoras,
incluso si parecen drásticas. Como Scrum Master, deberé recopilar todos estos
problemas y posibles soluciones, y analizar con el equipo qué se puede mejorar y
cómo. En caso de encontrar más problemas individuales, consideraré trabajar con
los miembros que generen estos problemas. En caso de encontrar más problemas
grupales, fomentar las retrospectivas va a ayudar a mejorar semana a semana
todos estos problemas.
\\
Finalmente, creo que también es necesario el componente humano. Como ya comenté
en la sesión 7, en el ejercicio de solucionar los problemas de los bomberos, soy
bastante partidario de los team events y de generar equipo fuera del ambiente de
trabajo. Con estos eventos no espero que salgan relaciones de por vida, pero sí
rebajar la tensión que pueda haber ente miembors y fomentar la confianza. Aunque
en una sesión de coaching puede ser complicado, quizás intentaría encontrar un
momento para hacer una partida de billar o algo similar, fuera del trabajo, para
tener está sesión.
\subsection{Problema 3}
Suponiendo que como SM he sido capaz de encontrar ciertos patrones a los
problemas tras las sesiones individuales y después de estar en los eventos de
equipo como las dailies, considero que me deberé esfozar en fomentar el feedback
constructivo. Es decir, deberé facilitar o guiar diferentes sesiones, a lo largo
de cierto tiempo, donde los miembros del equipo puedan expresarse y
conjuntamente darse cuenta de los problemas que existen, y proponer soluciones.
\\
Estás mejoras van a verse directamente reflejadas en la comunicación,
cooperación y proactividad de los miembros del equipo. UNa vez solucionados
estos problemas, ejercicios de establecimiento de objetivos globales o a nivel
de equipo van a ser más fáciles. Creo que, si no hay confianza, colaboración o
intención de cooperación, no es posible establecer objetivos reales a nivel de
equipo.
\\
Establecer objetivos reales es tan importante como la revisión de estos.
Objetivos que no son revisados o actualizados, terminan por perder importancia a
nivel de equipo, a demás de disminuir la motivación y el setimiento de
implicación. En las siguiente retrospectivas, el equipo deberá dedicar parte de
la sesión a dicha revisión y valoración. Es importante generar un espacio
abierto donde cuantos más integrantes interactúen, mejor.
\end{document}
